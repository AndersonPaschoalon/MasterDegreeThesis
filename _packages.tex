\documentclass[12pt,oneside,a4paper,brazil,english,sumario=tradicional,]{abntex2}
%\documentclass[draft]{abntex2}
%%%%%%%%%%%%%%%%%%%%%%%%%%%%%%%%%%%
\usepackage[hyphenbreaks]{breakurl}
\usepackage{booktabs}
\usepackage{minted}
\usepackage{morewrites}
%\DisemulatePackage{index}
\usepackage[table]{xcolor}
%\usepackage[subfigure]{tocloft} 
\usepackage{subfig} 
\usepackage{nth}
\usepackage{pifont}
%%%%%%%%%%%%%%%%%%%%%%%%%%%%%%%%%%%
\usepackage{cmap}
\usepackage{mathptmx}
%\usepackage{lmodern}
%\usepackage{helvet}
%\renewcommand{\familydefault}{\sfdefault}
%--------------
%\renewcommand{\rmdefault}{phv} % Arial
%\renewcommand{\sfdefault}{phv} % Arial	
\usepackage[T1]{fontenc}
%\usepackage{uarial}
%\renewcommand{\familydefault}{\sfdefault}
%\usepackage{blindtext}
\usepackage[utf8]{inputenc}
\usepackage{lastpage}
\usepackage{indentfirst}
\usepackage{framed}
\usepackage{color}
\usepackage{graphicx}
\usepackage{svg}
\usepackage{amsfonts}
\usepackage{tcolorbox}
\renewcommand{\thepage}{\roman{page}}
\usepackage{hyperref}
\usepackage{epstopdf}
\usepackage[referable]{threeparttablex}
\usepackage{lipsum}
\usepackage{blindtext}
\usepackage{caption}
%\usepackage{subcaption}
\usepackage{bbm}
%\usepackage[chapter]{algorithm}
%\usepackage{algorithmic}
\usepackage{multirow}
\usepackage{rotating}
\usepackage{eurosym}
\usepackage{pdfpages}
\usepackage{threeparttable}
\usepackage{rotating}
\usepackage[acronym,toc]{glossaries}
%\makeglossaries
%%% PROJETO

\newacronym{SIMITAR}{SIMITAR}{SnIffing, ModellIng, and TrAffic geneRation}

\newacronym{CDT}{CDT}{Compact Trace Descriptor}

\newacronym{flowID}{flowID}{Flow Identifier}

%% TECNOLOGIAS
\newacronym{SQL}{SQL}{Structured Query Language}

\newacronym{VNF}{VNF}{Virtualized Network Function}
 
\newacronym{NFV}{NVF}{Network Function Virtualization}
 
\newacronym{SDN}{SDN}{Software Defined Networking}

\newacronym{IoT}{IoT}{Internet of Things}

\newacronym{M2M}{M2M}{Machine to Machine}

\newacronym{DUT}{DUT}{Device Under Test}

\newacronym{I/O}{I/O}{Input/Output }

\newacronym{API}{API}{Application programming interface}

\newacronym{UML}{UML}{Unified Modeling Language}

\newacronym{GNU}{GNU}{GNU's Not Unix!}

\newacronym{NIC}{NIC}{Network Interface Card}

\newacronym{MTU}{MTU}{Maximum transmission unit}

\newacronym{QoS}{QoS}{Quality of service}

\newacronym{QoE}{QoE}{Quality of service}

\newacronym{WSA}{WSA}{Wavelet-based scaling analysis}

\newacronym{WMA}{WMA}{Wavelet multi-resolution energy analysis}

\newacronym{QQplot}{QQplot}{Quantile-quantile plot }

\newacronym{RTT}{RTT}{Round Trip Time}

\newacronym{FNV}{FNV}{Fowler-Noll-Vo}

\newacronym{ACK}{ACK}{Acknowledge}

\newacronym{SYN}{SYN}{Synchronize}

\newacronym{XML}{XML}{Extensible Markup Language}

\newacronym{CLI}{CLI}{Command Line Interface}

\newacronym{GUI}{GUI}{Graphical User Interface}

\newacronym{GAN}{GAN}{Generative adversarial network}

\newacronym{LAN}{LAN}{Local Area Network}

\newacronym{WAN}{WAN}{Wide area network}

\newacronym{KNI}{KNI}{Kernel NIC Interface}

\newacronym{NOS}{NOS}{Network Operational System}

\newacronym{IT}{IT}{Information Technology}

\newacronym{MANO}{MANO}{Management and Orchestration}

\newacronym{NFVI}{NFVI}{NFV Infrastructure}

\newacronym{NAT}{NAT}{Network Address Translation}

\newacronym{VLAN}{VLAN}{Virtual LAN}

\newacronym{SCTP}{SCTP}{Stream Control Transmission Protocol}

\newacronym{FPGA}{FPGA}{Field Programmable Gate Array}

\newacronym{NetFPGA}{NetFPGA}{Network FPGA}

% PROTOCOLOS

\newacronym{TCP}{TCP}{Transmission Control Protocol}

\newacronym{UDP}{UDP}{User Datagram Protocol}

\newacronym{IP}{IP}{Internet Protocol}

\newacronym{IPv4}{IPv4}{Internet Protocol Version 4}

\newacronym{IPv6}{IPv6}{Internet Protocol Version 6}

\newacronym{ICMP}{ICMP}{Internet Control Message Protocol}

\newacronym{IPsec}{IPsec}{Protocolo de Segurança IP}

\newacronym{ICMPv6}{ICMPv6}{Internet Control Message Protocol Version 6}

\newacronym{ARP}{ARP}{Address Resolution Protocol}

\newacronym{MAC}{MAC}{Media Access Control}

\newacronym{BGP}{BGP}{Border Gateway Protocol}

\newacronym{DHCP}{DHCP}{Dynamic Host Configuration  Protocol}

\newacronym{HTTP}{HTTP}{Hypertext Transfer Protocol}

\newacronym{HTTPS}{HTTPS}{Hyper Text Transfer Protocol Secure}

\newacronym{FTP}{FTP}{File Transfer Protocol}

\newacronym{TACACS}{TACACS}{Terminal Access Controller Access-Control System}

\newacronym{SSH}{SSH}{Secure Shell}

\newacronym{DNS}{DNS}{Domain Name System}

\newacronym{SNMP}{SNMP}{Simple Network Management Protocol}

% MATH

\newacronym{BIC}{BIC}{Bayesian information criterion}

\newacronym{AIC}{AIC}{Akaike information criterion}

\newacronym{PDF}{PDF}{Probability Density Function}

\newacronym{CDF}{CDF}{Cumulative Distribution Function}

\newacronym{LRD}{LRD}{Long-Range Dependence}

\newacronym{PT-MMPP}{PT-MMPP}{Power-tail Markov-Modulated}

\newacronym{MDL}{MDL}{Minimum Description Length}

\newacronym{nMDL}{nMDL}{Normalized Minimum Description Length}

\newacronym{AICc}{AICc}{Akaike's Information Criterion Corrected}

\newacronym{DIC}{DIC}{Deviance Information Criterion}

%\newacronym{}{}{}




%\usepackage[USenglish,hyperpageref]{backref} %NÚMERO DA PÁGINA DE UMA CITAÇÃO APARECENDO NAS REFERÊNCIAS 
\usepackage[alf,abnt-etal-cite=2,abnt-etal-list=0,abnt-etal-text=emph]{abntex2cite}	% 
\usepackage{unicamp}
%\graphicspath{{./eps/}}
%\DeclareGraphicsExtensions{.eps}
\newcommand{\mb}[1]{\mathbf{#1}}
\newtheorem{mydef}{Defini\c{c}\~{a}o}[chapter]
\newtheorem{lemm}{Lema}[chapter]
\newtheorem{theorem}{Teorema}[chapter]
%\floatname{algorithm}{Pseudoc\'{o}digo}
%\renewcommand{\listalgorithmname}{Lista de Pseudoc\'{o}digos}

%New packages-----------------------------------
%\usepackage[table,xcdraw]{xcolor}
\usepackage{blindtext, rotating}
\usepackage{comment}
\usepackage{breakurl}
\usepackage{multirow}
\usepackage{mathtools}
\usepackage[colorinlistoftodos]{todonotes}
\usepackage{amssymb}
\usepackage{cite}
\usepackage{hyperref}
\usepackage{url}
\usepackage[T1]{fontenc}
\usepackage{xcolor}
%\usepackage{subfig}
%\usepackage{titling}
\usepackage{pdfpages}
\usepackage{environ}
\usepackage{setspace}
\usepackage{minted}
\usepackage{tcolorbox}
%Trecho de código para realizar a listagem de código fonte
% Definindo novas cores
\definecolor{verde}{rgb}{0,0.5,0}
% Configurando layout para mostrar codigos C++
\usepackage{listings}
\lstset{
  language=C++,
  basicstyle=\ttfamily\small,
  keywordstyle=\color{blue},
  stringstyle=\color{verde},
  commentstyle=\color{red},
  extendedchars=true,
  showspaces=false,
  showstringspaces=false,
  numbers=left,
  numberstyle=\tiny,
  breaklines=true,
  backgroundcolor=\color{green!10},
  breakautoindent=true,
  captionpos=b,
  xleftmargin=0pt,
}

\usepackage{tabularx}
\usepackage{titlesec}% http://ctan.org/pkg/titlesec
\usepackage{makecell}
\usepackage{footnote}
\usepackage{imakeidx}
\usepackage{afterpage}
\makeindex[columns=3, title=Alphabetical Index]

%\makesavenoteenv{tabular}


%\renewcommand{\backrefpagesname}{Citado na(s) p\'{a}gina(s):~}
%\renewcommand{\backref}{}
%\renewcommand*{\backrefalt}[4]{
%	\ifcase #1 %
%		Nenhuma cita\c{c}\~{a}o no texto.%
%	\or
%		Citado na p\'{a}gina #2.%
%	\else
%		Citado #1 vezes nas p\'{a}ginas #2.%
%	\fi}%

\definecolor{blue}{RGB}{41,5,195}
\makeatletter
\hypersetup{
     	%pagebackref=true,
		pdftitle={\@title},
		pdfauthor={\@author},
    	pdfsubject={\imprimirpreambulo},
	    pdfcreator={LaTeX with abnTeX2},
		pdfkeywords={abnt}{latex}{abntex}{abntex2}{trabalho acad\^{e}mico},
		hidelinks,					% desabilita as bordas dos links
		colorlinks=false,       	% false: boxed links; true: colored links
    	linkcolor=blue,          	% color of internal links
    	citecolor=blue,        		% color of links to bibliography
    	filecolor=magenta,      	% color of file links
		urlcolor=blue,
%		linkbordercolor={1 1 1},	% set to white
		bookmarksdepth=4
}
\makeatother
\setlength{\parindent}{2cm}

% Controle do espa\c{c}amento entre um par\'{a}grafo e outro:
\setlength{\parskip}{0.2cm}

\orientador{Prof. Dr. Christian Rodolfo Esteve Rothenberg}
\instituicao{%
    UNIVERSIDADE ESTADUAL DE CAMPINAS
    \par
    Faculdade de Engenharia El\'{e}trica e de Computa\c{c}\~{a}o	
    }
\tipotrabalho{Disserta\c{c}\~{a}o (Mestrado)}
\preambulo{Dissertation presented to the Faculty of Electrical and Computer Engineering of the University of Campinas in partial fulfillment of the requirements for the degree of Master in Electrical Engineering, in the area of Computer Engineering. \vspace{0.4cm} \\
Disserta\c{c}\~{a}o apresentada à Faculdade de Engenharia Elétrica e Computa\c{c}\~{a}o da Universidade Estadual de Campinas como parte dos requisitos exigidos para a obten\c{c}\~{a}o do título de Mestre em Engenharia Eletrica, na Àrea de Engenharia de Computa\c{c}\~{a}o.}
% --- 

%\usepackage[english]{babel}
%\usepackage[utf8]{inputenc}
\usepackage{amsmath}
\usepackage{amsfonts}
\usepackage{graphicx}
\usepackage[colorinlistoftodos]{todonotes}
\usepackage{algorithm}
\usepackage{algpseudocode}



\chapter{Traffic Generators Survey}
\label{ap:traffic-gen-survey}


%%%%%%%%%%%%%%%%%%%%%%%%%%%%%%%%%%%%%%%%%%%%%%%%%%%%%%%%%%%%%%%%%%%%%%%%%%%%%%%%
\section{Traffic generator tools}


In this section we will presented a short review of many open-source tools available for synthetic traffic generation and benchmark. The goal in this section is to present both the most mentioned tools in the literature, and the most recent and advanced ones. On tables \label{tab:packet-level-tg}, \label{tab:multi-level-tg}, \label{tab:app-level-tg}, and \label{tab:replay-tg} is presented a survey of of the main features of such tools, such as support for Operational systems, protocols, stochastic functions available for traffic generation, and traffic generator class. Some free, but not open-source traffic generators are listed as well. 

Before present our survey, we will refer to some tools mentioned in literature, but we couldn't found source code and manual. \textbf{BRUNO}\cite{bruno-paper} is traffic generator implemented aiming performance and accuracy on timings. It has many configurable parameters that allow emulation of many web server scenarios. \textbf{Divide and conquer}\cite{validate-trafficgen}: Divide and conquer is replay engine that works in a distributed manner. It is able to split traces among multiple commodity PCs, and reply packets, to produce realistic traffic. Some others mentioned tools\cite{web-ditg} we weren't able to find any reference of available features are: \textbf{UDPgen}, \textbf{Network Traffic Generator}, \textbf{Packet Shell}, \textbf{Real-Time Voice Traffic Generator}, \textbf{PIM-SM Protocol Independent Multicast Packet Generator}, \textbf{TTCP}, \textbf{SPAK, Packet Generator}, \textbf{TfGen}, \textbf{TrafGen} and \textbf{Mtools}. In the table ~\ref{tab:traffic-gen-links} we present an updated list of links for download.

\subsection{Traffic Generators - Feature Survey}


%%%%%%%%%%%%%%%%%%%%%%%%%%%%%%%%%%%%%%%%%%%%%%%%%%%%%%%%%%%%%%%%%%%%%%%%%%%%%%%%%
% PACKET-LEVEL TGS
%%%%%%%%%%%%%%%%%%%%%%%%%%%%%%%%%%%%%%%%%%%%%%%%%%%%%%%%%%%%%%%%%%%%%%%%%%%%%%%%%
\begin{table}[]
\centering
\caption{Summary of packet-level traffic generators.}
\scalebox{0.7}{
\begin{tabular}{ccccc}
\hline
\rowcolor[HTML]{9B9B9B} 
\multicolumn{5}{c}{\cellcolor[HTML]{9B9B9B}\textbf{Packet-level Traffic Generators}} \\
\rowcolor[HTML]{9B9B9B} 
\textbf{\begin{tabular}[c]{@{}c@{}}Traffic\\ Generator\end{tabular}}    & \textbf{Operating System}                                                                                                      & \textbf{\begin{tabular}[c]{@{}c@{}}Network \\ Protocols\end{tabular}}                                                                                                                                                                               & \textbf{\begin{tabular}[c]{@{}c@{}}Available \\ stochastic distributions\end{tabular}}                                     & \textbf{Interface}                                                                         \\ \hline
\textbf{D-ITG}                                                                   & \begin{tabular}[c]{@{}c@{}}Linux, Windows, \\ Linux Familiar, \\ Montavista, Snapgear\end{tabular}                             & \begin{tabular}[c]{@{}c@{}}IPv4-6, ICMP, TCP\\ UDP, DCCP, SCTP\end{tabular}                                                                                                                                                                         & \begin{tabular}[c]{@{}c@{}}constant, uniform, \\ exponential, pareto, \\ cauchy, normal,  \\ poisson, gamma\end{tabular}   & \begin{tabular}[c]{@{}c@{}}CLI, \\ Script, \\ API\end{tabular}                             \\
\rowcolor[HTML]{C0C0C0} 
{\color[HTML]{000000} Ostinato}                                         & {\color[HTML]{000000} \begin{tabular}[c]{@{}c@{}}Linux, Windows, \\ FreeBDS\end{tabular}}                                      & {\color[HTML]{000000} \begin{tabular}[c]{@{}c@{}}Ethernet/802.3/LLC, \\ SNAP;  VLAN, (with QinQ); \\ ARP,  IPv4-6-Tunnelling;\\  TCP, UDP,  ICMPv4, \\ ICMPv6,  IGMP,  MLD; \\ HTTP,  SIP, RTSP,  NNTP,  \\ custom  protocol,  etc...\end{tabular}} & {\color[HTML]{000000} constant}                                                                                            & {\color[HTML]{000000} \begin{tabular}[c]{@{}c@{}}GUI,\\ CLI,\\ script,\\ API\end{tabular}} \\
PackETH                                                                 & \begin{tabular}[c]{@{}c@{}}Linux, MacOS,\\ Windows\end{tabular}                                                                & \begin{tabular}[c]{@{}c@{}}Ehernet II, ethernet 802.3, \\ 802.1q,  QinQ, ARP,\\  IPv4-6,  UDP,  TCP, ICMP, \\ ICMPv6, IGMP\end{tabular}                                                                                                             & constant                                                                                                                   & CLI, GUI                                                                                   \\
\rowcolor[HTML]{C0C0C0} 
{\color[HTML]{000000} Seagull}                                          & {\color[HTML]{000000} Linux, Windows}                                                                                          & {\color[HTML]{000000} \begin{tabular}[c]{@{}c@{}}IPv4-6, UDP, TCP, SCTP, \\ SSL/TLS and SS7/TCAP. \\ custom protocol\end{tabular}}                                                                                                                  & {\color[HTML]{000000} \begin{tabular}[c]{@{}c@{}}constant, \\ poisson\end{tabular}}                                        & {\color[HTML]{000000} CLI, GUI}                                                            \\
Iperf                                                                   & \begin{tabular}[c]{@{}c@{}}Windows, Linux, \\ Android, MacOS X,\\ FreeBSD, OpenBSD,\\ NetBSD, VxWorks, \\ Solaris\end{tabular} & IPv4-6, UDP, TCP, SCTP                                                                                                                                                                                                                              & constant                                                                                                                   & CLI, API                                                                                   \\
\rowcolor[HTML]{C0C0C0} 
{\color[HTML]{000000} BRUTE}                                            & {\color[HTML]{000000} Linux}                                                                                                   & {\color[HTML]{000000} IPv4-6, UDP, TCP}                                                                                                                                                                                                             & {\color[HTML]{000000} \begin{tabular}[c]{@{}c@{}}constant, poisson, \\ trimodal, exponential\end{tabular}}                 & {\color[HTML]{000000} CLI, script}                                                         \\
SourcesOnOff                                                            & Linux                                                                                                                          & IPv4, TCP, UDP                                                                                                                                                                                                                                      & \begin{tabular}[c]{@{}c@{}}weibull, pareto, \\ exponential, normal\end{tabular}                                            & CLI                                                                                        \\
\rowcolor[HTML]{C0C0C0} 
{\color[HTML]{000000} TG}                                               & {\color[HTML]{000000} \begin{tabular}[c]{@{}c@{}}Linux, FreeBSD, \\ Solaris SunOS\end{tabular}}                                & {\color[HTML]{000000} IPv4, TCP, UDP}                                                                                                                                                                                                               & {\color[HTML]{000000} \begin{tabular}[c]{@{}c@{}}constant, uniform, \\ exponential\end{tabular}}                           & {\color[HTML]{000000} CLI}                                                                 \\
Mgen                                                                    & Linux(Unix), Windows                                                                                                           & IPv4-6, UDP, TCP, SINK                                                                                                                                                                                                                              & constant, exponential,                                                                                                     & CLI, Script                                                                                \\
\rowcolor[HTML]{C0C0C0} 
KUTE                                                                    & Linux 2.6                                                                                                                      & UDP                                                                                                                                                                                                                                                 & constant                                                                                                                   & kernel module                                                                              \\
\begin{tabular}[c]{@{}c@{}}RUDE \& \\ CRUDE\end{tabular}                & \begin{tabular}[c]{@{}c@{}}Linux, Solaris SunOS, \\ FreeBSD\end{tabular}                                                       & IPv4, UDP                                                                                                                                                                                                                                           & constant                                                                                                                   & CLI                                                                                        \\
\rowcolor[HTML]{C0C0C0} 
NetSpec                                                                 & Linux                                                                                                                          & IPv4,UDP, TCP                                                                                                                                                                                                                                       & \begin{tabular}[c]{@{}c@{}}uniform, normal, log-normal, \\ exponential, poisson, \\ geometric,  pareto, gamma\end{tabular} & script                                                                                     \\
Nping                                                                   & Windows, Linux, Mac OS X                                                                                                       & \begin{tabular}[c]{@{}c@{}}TCP, UDP, ICMP, \\ IPv4-6, ARP\end{tabular}                                                                                                                                                                              & constant                                                                                                                   & CLI                                                                                        \\
\rowcolor[HTML]{C0C0C0} 
MoonGen                                                                 & Linux                                                                                                                          & \begin{tabular}[c]{@{}c@{}}IPv4-6, IPsec,\\  ICMP, UDP, TCP\end{tabular}                                                                                                                                                                            & constant, poisson                                                                                                          & scipt API                                                                                  \\
\begin{tabular}[c]{@{}c@{}}Dpdk \\ Pktgen\end{tabular}                  & Linux                                                                                                                          & \begin{tabular}[c]{@{}c@{}}IPv4, IPv6, ARP, \\ ICMP, TCP, UDP\end{tabular}                                                                                                                                                                          & constant                                                                                                                   & CLI, script API                                                                            \\
\rowcolor[HTML]{C0C0C0} 
LegoTG                                                                  & Linux                                                                                                                          & (depend on underlying tool)                                                                                                                                                                                                                         & (depend on underlying tool)                                                                                                & CLI, script                                                                                \\
\begin{tabular}[c]{@{}c@{}}gen\_send/\\ gen\_recv\end{tabular}          & \begin{tabular}[c]{@{}c@{}}Solaris, FreeBSD, \\ AIX4.1, Linux\end{tabular}                                                     & UDP                                                                                                                                                                                                                                                 & constant                                                                                                                   & CLI                                                                                        \\
\rowcolor[HTML]{C0C0C0} 
mxtraf                                                                  & Linux                                                                                                                          & TCP, UDP, IPv4                                                                                                                                                                                                                                      & constant                                                                                                                   & GUI, script                                                                                \\
\begin{tabular}[c]{@{}c@{}}Jigs Traffic \\ Generator (JTG)\end{tabular} & Linux                                                                                                                          & TCP, UDP, IPv4-6                                                                                                                                                                                                                                    & constant                                                                                                                   & CLI                                                                                        \\ \hline
\end{tabular}
}
\label{tab:packet-level-tg}
\end{table}


%%%%%%%%%%%%%%%%%%%%%%%%%%%%%%%%%%%%%%%%%%%%%%%%%%%%%%%%%%%%%%%%%%%%%%%%%%%%%%%%%
% MULTI/FLOW-LEVEL TGS
%%%%%%%%%%%%%%%%%%%%%%%%%%%%%%%%%%%%%%%%%%%%%%%%%%%%%%%%%%%%%%%%%%%%%%%%%%%%%%%%%
\begin{table}[]
\centering
\caption{Summary of multi-level and flow-level traffic generators.}
\scalebox{0.7}{
\begin{tabular}{ccccc}
\hline
\rowcolor[HTML]{9B9B9B} 
\multicolumn{5}{c}{\cellcolor[HTML]{9B9B9B}\textbf{Flow and Multi-level Traffic Generators}}                                                                                                                                                                                                                                                                                                                   \\
\rowcolor[HTML]{9B9B9B} 
\textbf{\begin{tabular}[c]{@{}c@{}}Traffic\\ Generator\end{tabular}} & \textbf{Operating System}                                                                          & \textbf{\begin{tabular}[c]{@{}c@{}}Network\\ Protocols\end{tabular}}                     & \textbf{Model}                                                                                             & \textbf{Interface}         \\ \hline
Swing                                                                & Linux                                                                                              & \begin{tabular}[c]{@{}c@{}}IPv4, TCP, UDP,\\ HTTP, NAPSTER,\\ NNTP and SMTP\end{tabular} & \begin{tabular}[c]{@{}c@{}}Multi-level \\ auto-configurable \\ Ethernet\end{tabular}                       & CLI                        \\
\rowcolor[HTML]{C0C0C0} 
{\color[HTML]{000000} Harpoon}                                       & {\color[HTML]{000000} \begin{tabular}[c]{@{}c@{}}FreeBSD, Linux, \\ MacOS X, Solaris\end{tabular}} & {\color[HTML]{000000} \begin{tabular}[c]{@{}c@{}}TCP, UDP, IPv4,\\ IPv6\end{tabular}}    & {\color[HTML]{000000} \begin{tabular}[c]{@{}c@{}}Flow-level \\ auto-configurable \\ Ethernet\end{tabular}} & {\color[HTML]{000000} CLI} \\
LiTGen                                                               & -                                                                                                  & -                                                                                        & Multi-level Wifi                                                                                           & -                          \\
\rowcolor[HTML]{C0C0C0} 
EAR                                                                  & Linu                                                                                               & \begin{tabular}[c]{@{}c@{}}IEEE 802.11, ICMP, UDP, \\ TCP, TFTP, Telnet\end{tabular}     & \begin{tabular}[c]{@{}c@{}}"Event Reproduction Ratio" techinique - \\ wireless IEEE 802.11\end{tabular}    & -                          \\ \hline
\end{tabular}
}
\label{tab:multi-level-tg}
\end{table}



%%%%%%%%%%%%%%%%%%%%%%%%%%%%%%%%%%%%%%%%%%%%%%%%%%%%%%%%%%%%%%%%%%%%%%%%%%%%%%%%%
% APPLICATION-LEVEL TGS
%%%%%%%%%%%%%%%%%%%%%%%%%%%%%%%%%%%%%%%%%%%%%%%%%%%%%%%%%%%%%%%%%%%%%%%%%%%%%%%%%
\begin{table}[]
\centering
\caption{Summary of application-level traffic generators. }
\scalebox{0.7}{
\begin{tabular}{cccc}
\hline
\rowcolor[HTML]{9B9B9B} 
\multicolumn{4}{c}{\cellcolor[HTML]{9B9B9B}Application-level Traffic Generators}                                                                                                                                                                                                                              \\ \hline
\rowcolor[HTML]{9B9B9B} 
\begin{tabular}[c]{@{}c@{}}Traffic\\ Generator\end{tabular} & Operating System                                                                                & Model                                                                                                                             & Interface \\ \hline
GenSyn                                                      & Java Virtual Machine                                                                            & User-behavior emulation                                                                                                           & GUI       \\
\rowcolor[HTML]{C0C0C0} 
D-ITG                                                       & \begin{tabular}[c]{@{}c@{}}Linux, Windows, Linux \\ Familiar, Montavista, Snapgear\end{tabular} & \begin{tabular}[c]{@{}c@{}}Telnet, DNS, Quake3,\\ CounterStrike (active and inactive), \\ VoIP (G.711, G.729, G.723)\end{tabular} & CLI       \\
Surge                                                       & Linux                                                                                           & Client/Server                                                                                                                     & CLI       \\
\rowcolor[HTML]{C0C0C0} 
Httperf                                                     & Linux                                                                                           & HTTP/1.0, HTTP/1.1                                                                                                                & CLI       \\
VoIP Traffic Generator                                      & -                                                                                               & VoIP                                                                                                                              & CLI       \\
\rowcolor[HTML]{C0C0C0} 
ParaSynTG                                                   & -                                                                                               & HTTP workload properties                                                                                                          & CLI       \\
NetSpec                                                     & LInux                                                                                           & \begin{tabular}[c]{@{}c@{}}HTTP, FTP, Telnet, Mpev video, \\ voice and video teleconference\end{tabular}                          & CLI       \\ \hline
\end{tabular}
}
\label{tab:app-level-tg}
\end{table}


%%%%%%%%%%%%%%%%%%%%%%%%%%%%%%%%%%%%%%%%%%%%%%%%%%%%%%%%%%%%%%%%%%%%%%%%%%%%%%%%%
% REPLAY ENGINES
%%%%%%%%%%%%%%%%%%%%%%%%%%%%%%%%%%%%%%%%%%%%%%%%%%%%%%%%%%%%%%%%%%%%%%%%%%%%%%%%%
\begin{table}[]
\centering
\caption{Summary of replay-engines traffic generators.}
\scalebox{0.7}{
\begin{tabular}{ccc}
\hline
\rowcolor[HTML]{9B9B9B} 
\multicolumn{3}{c}{\cellcolor[HTML]{9B9B9B}\textbf{Replay-Engines Traffic Generators}}                                                                           \\
\rowcolor[HTML]{9B9B9B} 
\textbf{\begin{tabular}[c]{@{}c@{}}Traffic\\ Generator\end{tabular}} & \textbf{Operating System}                                       & \textbf{Implementation} \\ \hline
Ostinato                                                             & Linux, Windows, FreeBDS                                         & Software-only           \\
\rowcolor[HTML]{C0C0C0} 
PackETH                                                              & \begin{tabular}[c]{@{}c@{}}Linux, MacOS,\\ Windows\end{tabular} & Software-only           \\
BRUNO                                                                & Linux                                                           & hardware-dependent      \\
\rowcolor[HTML]{C0C0C0} 
TCPReplay                                                            & Linux                                                           & Software-only           \\
TCPivo                                                               & Linux                                                           & Software-only           \\
\rowcolor[HTML]{C0C0C0} 
\begin{tabular}[c]{@{}c@{}}NetFPGA \\ PacketGenerator\end{tabular}   & Linux                                                           & Hardware                \\
\begin{tabular}[c]{@{}c@{}}NetFPGA \\ Caliper\end{tabular}           & Linux                                                           & Hardware                \\
\rowcolor[HTML]{C0C0C0} 
\begin{tabular}[c]{@{}c@{}}NetFPGA \\ OSTN\end{tabular}              & Linux                                                           & Hardware                \\
\textbf{MoonGen}                                                     & Linux                                                           & Hardware-dependent      \\
\rowcolor[HTML]{C0C0C0} 
\textbf{DPDK Pktgen}                                                 & Linux                                                           & Hardware-dependent      \\
\textbf{NFPA}                                                        & Linux                                                           & hardware-dependent      \\ \hline
\end{tabular}
}
\label{tab:replay-tg}
\end{table}



%%%%%%%%%%%%%%%%%%%%%%%%%%%%%%%%%%%%%%%%%%%%%%%%%%%%%%%%%%%%%%%%%%%%%%%%%%%%%%%%
\subsection{Packet-level traffic generators}


\textbf{D-ITG}\cite{ditg-paper}\cite{web-ditg}: D-ITG(Distributed Internet Traffic Generator) is a platform capable to produce IPv4 and IPv6 traffic defined by IDT and PS probabilistic distributions such as constant, uniform, Pareto, Cauch, Normal, Poisson, Gamma, Weibull, and On/Off; both configurable and pre-defined for many applications, from Telnet, through online games. It provide many flow-level options of customization, like duration, start delay and number of packets, support to many link-layer and transport-layer protocols, options, sources and destinations addresses/ports. If has support for NAT traversal, so it is possible to make experiments between two different networks separated by the cloud. D-ITG can be also used to measure packet loss, jitter, and throughput. D-ITG may be used through a CLI, scripts, or a C API, that can be use to create applications and remotely control another hosts through a daemon. 


\textbf{Ostinato}\cite{web-ostinato}: Ostinato is a packet crafter, network traffic generator and analyzer with a friendly GUI(“wireshark in reverse” as the documentation says) and a Python API. This tool permits craft and send packets of different protocols  at different rates. Support Server/Client communication, and a huge variety of protocols, from the link layer(such as 802.3 and VLAN) to the application layer (such HTTP and SIP). It is also possible to add any unimplemented protocols, through scripts defined by the user. 


\textbf{Seagull}\cite{wp-seagull}\cite{web-seagull}: Seagull is a traffic generator and test open-source tool, released by HP. It has support of many protocols, from link layer to application layer, and it support is easily extended, via XML dictionaries. As the documentation argues, the protocol extension flexibility is one of the main features. It support high speeds, and is reliable, being tested through hundreds of hours. It can also generate traffic using three statistical models: uniform(constant), best-effort and Poisson.


\textbf{BRUTE}\cite{web-brute}: Is a traffic generator that operates on the top of  Linux 2.4-6 and 2.6.x, not currently being supported on newer versions. It also support some stochastic functions(constant, poisson, trimodal) for departure time brust, and is able to simulate VoIP traffic. 

\textbf{PackETH} \cite{web-packeth}: PackETH is GUI and CLI stateless packet generator tool for ethernet.It support many adjustments of parameters, and many protocols as well, ans can set MAC addresses.


\textbf{Iperf}\cite{web-iperf}: Ipef is network traffic generator tools, designed for the measure of the maximum achievable bandwidth on IP networks, for both TCP and UDP traffic, bu can evaluate delay, windows size, and packet loss. It has a GUI interface, called Jperf\cite{web-jperf}. There is also a Java API, for automating tests\cite{jperf-git}. Support IPv4 and IPv6.


\textbf{NetPerf}\cite{web-netperf}: Netperf is a benchmark tools that can be used to measure performance of many types of networks, providing tests for both unidirectional throughput, and end-to-end latency. It has support for TCP, UDP and SCTP, both over IPv4 and IPv6.


\textbf{sourcesOnOff}\cite{sourcesonoff-paper} \cite{web-sourcesonoff}: sourcesOnOff is a recent traffic generator released on 2014, that aims to generate realistic synthetic traffic using probabilistic models to controll on and off time of traffic flows. As shown on the paper, it is able to guarantee self-similatiry, an has support to many probabilistic distributions for the on/off times: Weibull, Pareto, Exponential  and Gaussian. Supports TCP and UDP over IPv4. 


\textbf{TG}\cite{web-tg}: TG is a traffic generator that is able to generate and receive one-way packet streams transmitted from the UNIX user level process between source and traffic sink nodes. It is controlled by a simple specification language, that enables the craft of different lengths and interarrival times distributions, such as Constant, uniform, exponential and on/off(markov2).


\footnote{not open-source}\textbf{MGEN}\cite{web-mgen}: MGEN (Multi-Generator) is a traffic generator developed by the Naval Research Laboratory (NRL) PROTocol Engineering Advanced Networking (PROTEAN) Research Group. It uses as input can be used to emulate the traffic patterns of unicast and/or multicast UDP and TCP IP applications. It support many different types of stochastic functions, nominated periodic, Poisson, burst jitter and clone which are able to control inter departure times and packet size.


\textbf{KUTE}\cite{web-kute}: KUTE is a kernel level packet generator, designed to have a maximum performance traffic generator, and receiver mainly for use with Gigabit Ethernet. It works in the kernel level, sending packets as fast as possible, direct to the hardware driver, bypassing the stack. But KUTE works only on Linux 2.6, and has only be tested on Ethernet Hardware. Also, it only supports constant UDP traffic. 


\textbf{RUDE \& CRUDE}\cite{web-rude-crude}: RUDE(Real-time UDP Data Emitter) and CRUDE(Collector for RUDE), are small and flexible programs which runs on user-level. It has a GUI called GRUDE.  Ut works only with UDP protocol.


\footnote{not open-source}\textbf{NetSpec}\cite{web-netspec}: NetSpec is tool designed to do network tests, as opposed to doing point to point testing. NetSpec provides a framework that allows a user to control multiple processes across multiple hosts from a central point of control, using daemons that implement traffic sources and sinks, along with measurement tools. Also, it is able to model many different traffic patterns and applications, such as host maximum rate, Constant Bit Rate (CBR), WWW, World Wide Web, FTP, File Transfer Protocol, telnet, MPEG video, voice and video teleconference. 


\textbf{Nping}\cite{web-nping}: active hosts, as a traffic generator for network stack stress testing, ARP poisoning, Denial of Service attacks, route tracing, etc. Nping CLI permits the users control over protocols headers.


\textbf{TCPreplay}\cite{web-tcpreplay}: TCPreplay is a user-level replay engine, that can use pcap files as input, and then forward, packets in a network interface. It can modify some header parameters as well.


\textbf{TCPivo}\cite{tcpivo-paper}\cite{web-tcpivo}: TCPivo is a high-speed traffic replay engine that is able to read traffic traces, and replay packets in a network interface, working at kernel level. It is not currently supported kernel versions greater than 2.6.



\textbf{NetFPGA PacketGenerator}\cite{web-netfpgapacketgenerator}: NetFPGA PacketGenerator is a hardware-based traffic generator and capture tool, build over the NetFPGA 1G, and open FPGA platform with 4 ethernet interfaces of 1 Gigabit of bandwidth each. It is a replay engine tool which uses as input \textit{pcap} files. It is able to accurately control the delay between the frames, with the default delay being the same in the pcap file. It is also able to capture packets, and report statistics of the traffic. 


\textbf{NetFPGA Caliper}\cite{web-caliper}:  is a hardware-based traffic generator, build on NetFPGA 1G, built over NetThreads platform, a FPGA microprocessor which support threads programing. Different form NetFPGA PacketGenerator, Caliper is able to produce live packets. It is written in C. 


\textbf{NetFPGA OSNT}\cite{web-osnt}: OSNT(Open Source Network Tester) is hardware based network traffic generator built over the NetFPGA 10G. As NetFPGA 1G, NetFPGA 10G is a FPGA platform with 4 ethernet interfaces, but with 10 Gigabits of bandwidth.  OSNT is a replay engine, and is loaded with pcap traces. 


\textbf{Dpdk Pktgen}\cite{web-dpdk-pktgen}: Pktgen is a traffic generator measurer built over DPDK. DPDK is a development kit, a set of libraries and drivers for fast packet processing. DPDK was designed to run on any processor, but has some limitation on terms of supported NICs, that can be found on its website.


\textbf{MoonGen}\cite{moongen-paper}\cite{web-moongen} : MoonGen is a scriptable high-speed packet generator built over DPDK and LuaJIT. It is able to send packets at 10 Gbit/s, even with 64 bytes packets on a single CPU core. MoonGen can achieve this rate even if each packet is modified by a Lua script. Also, it provides accurate timestamping and rate control. It is able to generate traffic using several protocols ( IPv4, IPv6, IPsec, ARP, ICMP, UDP, and  TCP), and can generate different inter-departure times, like a Poisson process and burst traffic.  

\textbf{gen\_send/gen\_recv}\cite{web-gensend-genrecv}: gen\_send and gen\_recv are simple UDP traffic generator applications. It uses UDP sockets, and  gen\_send is able to control features like desired data rate, packet size and inter packet time. 


\textbf{mxtraf}\cite{web-mxtraf}: mxtraf enables that a small number of hosts to saturate a network, with a tunable mixture of TCP and UDP traffic. 

\textbf{Jigs Traffic Generator (JTG)}\cite{web-jtg}: is a simple, accurate traffic generator. JTG process only sends one stream of traffic, and steam characteristics are defined only by command line arguments. It also supports IPv6. 


%\textbf{*Poisson Traffic Generator}:


%%%%%%%%%%%%%%%%%%%%%%%%%%%%%%%%%%%%%%%%%%%%%%%%%%%%%%%%%%%%%%%%%%%%%%%%%%%%%%%%
\subsection{Application-level/Special-scenarios traffic generators}


\footnote{not open-source}\textbf{ParaSynTG}\cite{parasyntg-paper}: application-level traffic generator configurable by input parameters, which considers most of the observes www traffic workload properties. 


\footnote{not open-source}\textbf{EAR}\cite{ear-paper}: traffic generator that uses a technique called “Event Reproduction Ratio”  to mimic wireless IEEE 802.11 protocol behavior.


\footnote{not open-source}\textbf{GenSyn}\cite{web-gensyn}: network traffic generator implemented in Java, that mimic TCP and UDP connections, based on user behavior. 


\textbf{Surge}\cite{surge-paper}: Surge is an application level workload generator which emulates a set of real users accessing a web server. It matches many empirical measurements of real traffic, like server file distribution, request size distribution, relative file popularity, idle periods of users and other characteristics. 

\textbf{Httperf}\cite{web-httperf}: Is an application lever traffic generator to measure web server performance. It uses the protocol HTTP (HTTP/1.0 and HTTP/1.1), and offer many types of workloads while keeping track of statistics related to the generated traffic. Its most basic operation is to generate a set of HTTP GET requests and measure the number of replies and response rate.  

\textbf{VoIP Traffic Generator}: it is a traffic generator written in Perl, and creates multiple streams of traffic,  aiming to simulate VoIP traffic.


%%%%%%%%%%%%%%%%%%%%%%%%%%%%%%%%%%%%%%%%%%%%%%%%%%%%%%%%%%%%%%%%%%%%%%%%%%%%%%%%
\subsection{Flow-level and multi-level traffic generators}

\textbf{Harpoon}\cite{harpoon-paper}: Harpoon is a flow based traffic generator, that is able to 
automatically extract form Netflow traces parameters, in order to generate flows that exhibit the same statistical characteristics measured before, including temporal and spatial characteristics. 

\textbf{Swing}\cite{swing-paper} \cite{web-swing}: Swing is a closed-loop multi-layer, and network responsive generator. It is able to read capture traces, and captures the packet interactions of many applications, being able to models distributions for user, application, and network behavior, stochastic and responsively. As mentioned in the previous section, Swing is able to model user behavior, REEs, connection, packets, and network. 


\footnote{not open-source}\textbf{LiTGen}(Lightweight Traffic Generator)\cite{litgen-paper} is a open-loop, multilevel traffic generator. It is able to model wireless network traffic in a peer user and application basis. This tool models the traffic in three different levels: packet level, object level (smaller parts of an application session), and session level.


%%%%%%%%%%%%%%%%%%%%%%%%%%%%%%%%%%%%%%%%%%%%%%%%%%%%%%%%%%%%%%%%%%%%%%%%%%%%%%%%
\subsection{Others traffic generation tools}


\textbf{NFPA}\cite{nfpa-paper}: NFPA is a benchmark tools based on DPDK Pkgen, specialized on executing and automatize performance measurements over network functions. It works being directly connected to an specific device under tests. It uses built-in and user defined traffic traces, and Lua scripts control and collect information of DPDK Pktgen. It has an command line and Web interface, and automatically plot the results.

\textbf{LegoTG}\cite{legotg-paper}: LegoTG is a modular framework for composing custom traffic generation. It aims to simplify the combination on the use of different traffic generators and modulators on different testbeds, automatizing the process of installation, execution, resource allocation and synchronization via a centralized orchestrator, which uses a software repository. It already has support to many tools, and to add support to new tools is necessary to add and edit two files, called TGblock, and ExFile.


\subsection{Traaffic Generators -- Repository Survey}

\sloppy
\begin{table}[ht!]
\centering
\sloppy
\caption{Links for the traffic generators repositories}
\label{tab:traffic-gen-links}
%\begin{tabular}{@{}ll@{}}
\begin{tabularx}{\textwidth}{@{}ll@{} p{10.0cm}}
\toprule
\sloppy
Traffic Generator       & Repository                                                                                                                                                                                                                \\ \midrule
D-ITG                   & \href{http://traffic.comics.unina.it/software/ITG/}{http://traffic.comics.unina.it/software/ITG/}                                                                                                                         \\
Ostinato                & \href{http://ostinato.org/}{http://ostinato.org/}                                                                                                                                                                         \\
Seagull                 & \href{http://gull.sourceforge.net/}{http://gull.sourceforge.net/}                                                                                                                                                         \\
BRUTE                   & \href{http://wwwtlc.iet.unipi.it/software/brute/ }{http://wwwtlc.iet.unipi.it/software/brute/ }                                                                                                                           \\
PackETH                 & \href{http://packeth.sourceforge.net/packeth/Home.html}{http://packeth.sourceforge.net/packeth/Home.html}                                                                                                                 \\
Iperf                   & \href{https://iperf.fr/}{https://iperf.fr/}                                                                                                                                                                               \\
NetPerf                 & \href{http://www.netperf.org/netperf/}{http://www.netperf.org/netperf/}                                                                                                                                                   \\
sourcesOnOff            & \href{http://www.recherche.enac.fr/~avaret/sourcesonoff}{http://www.recherche.enac.fr/~avaret/sourcesonoff}                                                                                                               \\
TG                      & \href{http://www.postel.org/tg/}{http://www.postel.org/tg/}                                                                                                                                                               \\
MGEN*                   & \href{http://www.nrl.navy.mil/itd/ncs/products/mgen }{http://www.nrl.navy.mil/itd/ncs/products/mgen }                                                                                                                     \\
KUTE                    & \href{http://caia.swin.edu.au/genius/tools/kute/}{http://caia.swin.edu.au/genius/tools/kute/}                                                                                                                             \\
RUDE \& CRUDE            & \href{http://rude.sourceforge.net/}{ http://rude.sourceforge.net/}                                                                                                                                                        \\
Pktgen                  & \href{http://www.linuxfoundation.org/collaborate/workgroups/networking/pktgen}{http://www.linuxfoundation.org/collaborate/workgroups/networking/pktgen}                                                                   \\
NetSpec                 & \href{http://www.ittc.ku.edu/netspec/}{http://www.ittc.ku.edu/netspec/}                                                                                                                                                   \\
Nping                   & \href{https://nmap.org/nping/ }{https://nmap.org/nping/}                                                                                                                                                                  \\
TCPreplay               & \href{http://tcpreplay.appneta.com/}{http://tcpreplay.appneta.com/}                                                                                                                                                       \\
TCPivo                  & \href{http://www.thefengs.com/wuchang/work/tcpivo/}{http://www.thefengs.com/wuchang/work/tcpivo/}                                                                                                                         \\
NetFPGA PacketGenerator & \href{https://github.com/NetFPGA/netfpga/wiki/PacketGenerator}{https://github.com/NetFPGA/netfpga/wiki/PacketGenerator}                                                                                                   \\
NetFPGA Caliper         & \href{https://github.com/NetFPGA/netfpga/wiki/PreciseTrafGen}{https://github.com/NetFPGA/netfpga/wiki/PreciseTrafGen}                                                                                                     \\
NetFPGA OSNT            & \href{https://github.com/NetFPGA/OSNT-Public/wiki/OSNT-Traffic-Generator}{https://github.com/NetFPGA/OSNT-Public/wiki/OSNT-Traffic-Generator}                                                                             \\
DPDK Pktgen             & \href{http://pktgen.readthedocs.io/en/latest/getting_started.html}{http://pktgen.readthedocs.io/en/latest/getting\_started.html}                                                                                          \\
MoonGen                 & \href{https://github.com/emmericp/MoonGen}{https://github.com/emmericp/MoonGen}                                                                                                                                           \\
gen\_send/gen\_recv       & \href{http://www.citi.umich.edu/projects/qbone/generator.html}{http://www.citi.umich.edu/projects/qbone/generator.html}                                                                                                   \\
mxtraf                  & \href{http://mxtraf.sourceforge.net/}{http://mxtraf.sourceforge.net/}                                                                                                                                                     \\
JTG                     & \begin{tabular}[c]{@{}l@{}}\href{https://sourceforge.net/projects/iperf/files/}{https://sourceforge.net/projects/iperf/files/}\\ \href{https://github.com/AgilData/jperf}{https://github.com/AgilData/jperf}\end{tabular} \\
GenSyn                  & \href{http://www.item.ntnu.no/people/personalpages/fac/poulh/gensyn}{http://www.item.ntnu.no/people/personalpages/fac/poulh/gensyn}                                                                                       \\
SURGE                   & \href{http://cs-www.bu.edu/faculty/crovella/surge_1.00a.tar.gz}{http://cs-www.bu.edu/faculty/crovella/surge\_1.00a.tar.gz}                                                                                                \\
Httperf                 & \href{https://linux.die.net/man/1/httperf}{https://linux.die.net/man/1/httperf}                                                                                                                                           \\
VoIP Traffic Generator  &    \href{https://sourceforge.net/projects/voiptg/}{https://sourceforge.net/projects/voiptg/}                                                                                                                                                                                                                       \\
Harpoon                 & \href{http://cs.colgate.edu/~jsommers/harpoon/}{http://cs.colgate.edu/~jsommers/harpoon/}                                                                                                                                 \\
Swing                   & \href{http://cseweb.ucsd.edu/~kvishwanath/Swing/}{http://cseweb.ucsd.edu/~kvishwanath/Swing/}                                                                                                                             \\
NFPA                    &                                                                                                                                                                                                                           \\
LegoTG                  &                                                                                                                                                                                                                           \\ \bottomrule
\end{tabularx}
%\end{tabular}
\end{table}



\section{Validation of Ethernet traffic generators: some use cases}


In this section we list some use cases of validation of Ethernet traffic generators. Our validation methods used on chapter~\ref{ch:modeling-evaluation} and~\ref{ch:validation} were based on them. We are going to present seven different study cases on validation of related traffic generators. They are Swing\cite{swing-paper}, Harpoon\cite{harpoon-paper}, D-ITG\cite{ditg-paper}, sourcesOnOff\cite{sourcesonoff-paper}, MoonGen\cite{moongen-paper}, LegoTG\cite{legotg-paper} and NFPA\cite{nfpa-paper}.



\subsubsection{Swing}
Swing\cite{swing-paper} is at present, one of the main references of realistic traffic generation.  The authors extracted bidirectional metrics from a network link of synthetic traces. Their goals were to get realism, responsiveness, and randomness.  They define realism as a trace that reflects the following characteristics of the original: 

\begin{itemize}
	\item Packet inter-arrival rate and burstiness across many time scales;
	\item Packet size distributions;
	\item Flow characteristics as arrival rate and length distributions;
	\item Destination IPs and port distributions.
\end{itemize}

The traffic generator uses a structural model the account interactions between many layers of the network stack. Each layer has many control variables, which is randomly generated by a stochastic process.  They begin the parameterization, classifying tcpdump\cite{web-libpcap} \textit{pcap}  files with the data, they are able to estimate parameters. 

They validate the results using publicly available traffic traces, from Mawi\cite{web-mawi} and CAIDA\cite{web-caida}. On the paper, the author focuses on these validation metrics:

\begin{itemize}
	\item Comparison of estimated parameters of the original and swing  generated traces;
	\item Comparison of aggregate and per-application bandwidth and packets per seconds ;
	\item QoS metrics such as two-way delay and loss rates;
	\item Scaling analysis, via Energy multiresolution energy analysis.
\end{itemize}


To the vast majority of the results, both original and swing traces results were close to each other. Thus,  Swing was able to match aggregate and burstiness metrics, per byte and per packet, across many timescales.  


%%%%%%%%%%%%%%%%%%%%%%%%%%%%%%%%%%%%%%%%%%%%%%%%%%%%%%%%%%%%%%%%%%%%%%%%%%%%%%%%
\subsubsection{Harpoon}

Harpoon\cite{harpoon-validation}\cite{harpoon-paper} is a traffic generator able to generate representative traffic at IP \textit{flow level}. It can generate TCP and IP with the same byte, packet, temporal and spatial characteristics measured at routers. Also, Harpoon is a self-configurable tool, since it automatically extracts parameters from network traces. It estimates some parameters from original traffic trace: file sizes, inter-connection times, source and destination IP addresses, and the number of active sessions. 

As proof of concept \cite{ harpoon-validation}, the authors compared statistics from original, and harpoon's generated traces. The two main types of comparisons: diurnal throughput, and for stochastic variable CDF and frequency distributions. Diurnal throughput refers to the average bandwidth variation within a day period.  In a usual network, during the day the bandwidth consumption is larger, and at night smaller. Also, they compared:

\begin{itemize}
	\item CDF of bytes transferred per 10 minutes
	\item CDF of packets transferred per 10 minutes
	\item CDF of inter-connection time
	\item CDF of file size
	\item CDF of flow arrivals per 1 hour
	\item Destination IP address frequency
\end{itemize}

In the end, they showed the differences in throughput evaluation of a Cisco 6509 switch/router using Harpoon and a constant rate traffic generator.  
Harpoon was proven able to give close CDFs, frequency and diurnal throughput plots compared to the original traces. Also, the results demonstrated that Harpoon provides a more variable load on routers, compared to constant rate traffic. It indicates the importance of using realistic traffic traces on the evaluation of equipment and technologies. 

%%%%%%%%%%%%%%%%%%%%%%%%%%%%%%%%%%%%%%%%%%%%%%%%%%%%%%%%%%%%%%%%%%%%%%%%%%%%%%%%
\subsubsection{D-ITG}

D-ITG\cite{ditg-paper} is a network traffic generator, with many configurable features. The tool provides a platform that meets many emerging requirements for a realistic traffic generation. For example, multi-platform, support of many protocols, distributed operation, sending/receiving flow scalability, generation models, and analytical model based generation high bit/packet rate. You can see different analytical and models and protocols supported by D-ITG at table ~\ref{tab:packet-level-tg}. 

We will focus on the evaluation of realism on analytical model parameterization.  It is a synthetic replication of a LAN party of eight players of Age of Mythology \footnote{\href{https://www.ageofempires.com/games/aom/}{https://www.ageofempires.com/games/aom/}}. They have captured traffic flows during the party. Then, they modeled its packet size and inter-packet time distributions. They show that the synthetic traffic and the analytical model have similar curves of packet size and inter-packet time; thus it can approximate the empirical data. Also, the mean and the standard deviation of the bit rate of the empirical and synthetic data are similar. 


%%%%%%%%%%%%%%%%%%%%%%%%%%%%%%%%%%%%%%%%%%%%%%%%%%%%%%%%%%%%%%%%%%%%%%%%%%%%%%%%
\subsubsection{sourcesOnOff}


Varet et al. \cite{sourcesonoff-paper} create an application implemented in C, called SourcesOnOff. It models the activity interval of packet trains using probabilistic distributions. To choose the best stochastic models, the authors have captured traffic traces using TCPdump. Then the developed tool can figure out what distribution (Weibull, Pareto, Exponential, Gaussian, etc.) fits better the original traffic traces. It uses the Bayesian Information Criterion (BIC) for distance assessment. It tests the smaller BIC for each distribution and selects it as the best choice. It ensured good correlation between the original and generated traces and self-similarity.

The validation methods used on sourcesOnOff are: 
\begin{itemize}
	\item Visual comparison between On time and Off time of the original trace and the stochastic fitting;
	\item QQplots, which aim to evaluate the correlation between inter-trains duration of real and generated traffic;
	\item Measurement of Autocorrelation\footnote{
		The autocorrelation functions measures the correlation between data samples $y_{t}$ and $y_{t + k}$, where $k =0, ..., K$, and the data sample $\{y\}$  is generated by a stochastic process.
		
		According to \cite{book-time-series-analysis}, the autocorrelation for a lag $k$ is:
		
		\begin{equation}
		r_{k} = \frac{c_{k}}{c_{0}}
		\end{equation}
		
		where 
		
		\begin{equation}
		c_{k} = \frac{1}{T - 1}\sum_{t = 1}^{T - k} (y_{t} - \bar{y})(y_{t+k} - \bar{y})
		\end{equation}
		
		and $c_{0}$ is the sample variance of the time series. 
	} of the measured throughput of the real and synthetic traffic;
	\item Hurst exponent computation of the real and the synthetic trace;
\end{itemize}

The results pointed to an excellent stochastic fitting, but better for On-time values. On the other hand, the correlation value of the QQplot was more significant on the Off time values (99.8\% versus 97.9\%). In the real and synthetic traces, the autocorrelation of the throughput remained between an upper limit of 5\%. Finally, the ratio between the evaluated Hurst exponent always remained smaller than 12\%.


%%%%%%%%%%%%%%%%%%%%%%%%%%%%%%%%%%%%%%%%%%%%%%%%%%%%%%%%%%%%%%%%%%%%%%%%%%%%%%%%
\subsubsection{MoonGen}

MoonGen\cite{moongen-paper} is a high-speed scriptable paper capable of saturating 10 GnE link with 64 bytes packets, using a single CPU core. The authors have built it over DPDK and LuaJit, enabling the user to have high flexibility on the crafting of each packet, through Lua scripts. It has multi-core support and runs on commodity server hardware. It can test latency with sub-microsecond precision and accuracy, using hardware timestamping of modern NICs cards. The Lua scripting API enable the implementation and high customization along with high-speed. This includes the controlling of packet sizes and control of inter-departure times.

The authors evaluated this traffic generator focused on throughput metrics, rather than others. Also, they have small packet sizes (64 bytes to 256), since the per-packet costs dominate. In their work, they were able to surpass 15 Gbit/s with an XL710 40 GbE NIC. Also, they achieve throughput values close to the line rate with packets of 128 bytes, and 2 CPU cores. 


%%%%%%%%%%%%%%%%%%%%%%%%%%%%%%%%%%%%%%%%%%%%%%%%%%%%%%%%%%%%%%%%%%%%%%%%%%%%%%%%
\subsubsection{LegoTG}

Bartlett et al.\cite{legotg-paper} implements a modular framework for composing custom traffic generation, called LegoTG. As argued by the authors (and by this present work), automation of many aspects of traffic generation is a desirable feature. The process of how to generate proper background traffic may become research by itself. Traffic generators available today offer a single model and a restricted set of features into a single code base, makes customization difficult. 

The primary purpose of their experiment is to show how LegoTG can generate background traffic, only. Also, it shows how realistic background traffic can influence research conclusions. The test chosen is one of the use cases proposed for Swing\cite{background-traffic-matter}, and it evaluates the error on bandwidth estimation of different measurement tools. It shows that LegoTG can provide secure and custom traffic generation.





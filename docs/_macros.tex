%%%%%%%%%%%%%%%%%%%%%%%%%%%%%%%%%%%%%%%%%%%%%%%%%%%%%%%%%%%%%%%%%%%%%%%%%%%%%%%%
\newcommand{\Author}{
Anderson dos Santos Paschoalon
}

%%%%%%%%%%%%%%%%%%%%%%%%%%%%%%%%%%%%%%%%%%%%%%%%%%%%%%%%%%%%%%%%%%%%%%%%%%%%%%%%
\newcommand{\Year}{
	2019
}

%%%%%%%%%%%%%%%%%%%%%%%%%%%%%%%%%%%%%%%%%%%%%%%%%%%%%%%%%%%%%%%%%%%%%%%%%%%%%%%%
\newcommand{\ThesisTitle}{
SIMITAR: Synthetic and Realistic Network Traffic Generation
}

%%%%%%%%%%%%%%%%%%%%%%%%%%%%%%%%%%%%%%%%%%%%%%%%%%%%%%%%%%%%%%%%%%%%%%%%%%%%%%%%
\newcommand{\TituloDaTese}{
SIMITAR: Geração de Tráfego de Rede Sintético e Realístico
}

%%%%%%%%%%%%%%%%%%%%%%%%%%%%%%%%%%%%%%%%%%%%%%%%%%%%%%%%%%%%%%%%%%%%%%%%%%%%%%%%
\newcommand{\AtaDeDefesa}{
\textbf{COMISSÃO JULGADORA - DISSERTAÇÃO DE MESTRADO}
\vspace{1cm}
\begin{flushleft}
\textbf{Candidato}: Anderson dos Santos Paschoalon \hspace{1cm}     RA: 083233 \\
\textbf{Data da Defesa}: 17/12/2018\\
\textbf{Título da Tese}: \\
``SIMITAR: SIMITAR: Synthetic and Realistic Network Traffic Generation''\\%english\\
``SIMITAR: Geração de Trafego de Rede Sintético e Realistico''%portuguese
\end{flushleft}
\vspace{0.2cm}
\begin{flushleft}Prof. Dr. Christian Rodolfo Esteve Rothenberg (Presidente, FEEC/UNICAMP)\\
Prof. Dr. Lee Luan Ling (FEEC/UNICAMP) - Membro Titular\\
Prof. Dr. Daniel Macêdo Batista (IME/USP) - Membro Titular
\end{flushleft}
\vspace{0.2cm} 
\begin{flushleft}Ata de defesa, com as respectivas assinaturas dos membros da Comissão Julgadora, encontra-se no processo de vida acadêmica do aluno. \end{flushleft}
}

%%%%%%%%%%%%%%%%%%%%%%%%%%%%%%%%%%%%%%%%%%%%%%%%%%%%%%%%%%%%%%%%%%%%%%%%%%%%%%%%
\newcommand{\Abstract}{

Realistic network traffic has a different impact compared to constant traffic generated by tools like Iperf, even with the same average bandwidth. Busty traffic may cause buffers overflows while constant traffic does not, and can decrease the measurement accuracy. The number of flows of workload may have an impact on flow-oriented nodes, such as SDN switches and controllers. In a scenario where software-defined networks will play an essential role in the future internet, a more in-depth validation of new technologies considering these aspects is crucial. Also, most of the open-source realistic traffic generator tools have the modeling layer coupled to the traffic generator, making a challenge update it to newer libraries and becoming them often outdated. Often most of the tools that support realistic traffic generation offer a large set of options to be configured but are not auto-configurable. So the production of actual realistic traffic is a challenging project by itself.
In this work, we more in-depth discuss this subject. As a final result for our research, we highlight two main contributions: a review of the available solutions in the literature and the propose of our traffic generator, called SIMITAR: SnIfing, ModellIng and TrAffic geneRation. 
This technology has a separated modeling framework from the traffic generator, being flow-oriented and auto-configurable. It creates and uses small traces descriptors as inputs - XML files that describe all features of the traffics. Currently, we may replicate with accuracy flow characteristics of all tested traffic, and the scaling features of some as well.
 We dedicated a particular focus on inter-packet times modeling, where we proposed a methodology based on information criteria for automating the process modeling and selection of the best model. We also proposed a validation method to measure the quality of choice.

\vspace{\onelineskip}

\noindent\textbf{Keywords}: traffic generators; network traffic modelling; burstier traffic; realistic traffic; \textit{pcap} file; packet sniffing; inter packet times; linear regression; gradient descendent; Cumulative Distribution Function (CDF); maximum likelihood; Akaike Information Criterion (AIC); Bayesian Information Criterion (BIC); packet trains;  Wavelet Multiresolution Analisis; Hurst Exponent.
}

%%%%%%%%%%%%%%%%%%%%%%%%%%%%%%%%%%%%%%%%%%%%%%%%%%%%%%%%%%%%%%%%%%%%%%%%%%%%%%%%
\newcommand{\Resumo}{
Um tráfego de rede realista tem um impacto diferente comparado a um tráfego constante gerado por ferramentas como o Iperf, mesmo com a mesma largura de banda média. Um tráfego em rajadas realisticas("\textit{burstier}") pode causar estouros de buffers enquanto  um tráfego constante de mesma largura média não; e pode diminuir a precisão da medição. O número de fluxos de carga de trabalho pode ter um impacto nos nós orientados a fluxo, como switches e controladores SDN. Em um cenário em que as redes definidas por software desempenharão um papel essencial na Internet futura, uma validação mais aprofundada das novas tecnologias, considerando esses aspectos, é crucial. Além disso, a maioria das ferramentas geradoras de tráfego realistas de código aberto tem a camada de modelagem acoplada ao gerador de pacotes, o que dificulta sua atualização para bibliotecas mais novas, tornando-as frequentemente desatualizadas. Por fim, a maioria das ferramentas \textit{open-source} que suportam a geração de tráfego realista oferece um grande conjunto de opções a serem configuradas, mas não são configuráveis automaticamente. Dessa forma a produção de um tráfego realista customizado é um projeto desafiador por si só.
  
Neste trabalho nos aprofundamos neste assunto. Como resultado final, para nossa pesquisa destacamos duas contribuições principais: uma investigação de revisão das soluções disponíveis na literatura e propomos nossas técnicas para modelagem de tráfego de rede.

Propomos nossa própria solução geradora de tráfego denominada SIMITAR: SnIffing, modelagem e TrAffic geneRation, que possui uma estrutura de modelagem separada do gerador de tráfego, que é orientada por fluxo e é configurável automaticamente. Ele cria e usa como entradas descritores de pequenos rastreios, arquivos XML que descrevem todos os recursos dos tráfegos. Atualmente, já podemos replicar com características de fluxo de precisão de todo o tráfego testado e os recursos de dimensionamento de alguns também.

Demos um enfoque especial na modelagem de tempos entre pacotes, onde propomos uma metodologia baseada em critérios de informação para automatizar a modelagem de processos e seleção do melhor modelo. Também propusemos um método de validação para medir a qualidade da escolha.

\vspace{\onelineskip}

\noindent\textbf{Keywords}: geradores de tráfego; modelagem de tráfego de rede;  tráfego em rajadas; tráfego realistico; arquivo \textit{pcap}; captura de pacotes; tempo entre pacotes; regressão linear; gradiente descendente; Função Distribuição Acumulada; máxima verossimilhança; Critério de informação de Akaike; Critério de informação Bayesiano; trem de pacotes;  Análise Wavelet de multiresolução; Expoente de Hurst.

}

%%%%%%%%%%%%%%%%%%%%%%%%%%%%%%%%%%%%%%%%%%%%%%%%%%%%%%%%%%%%%%%%%%%%%%%%%%%%%%%%
\newcommand{\Acknowledgements}{

First, I would like to thanks all who have helped me, directly or indirectly. Those who have inspired me to follow this path, those who have taught and helped me, and those whose just their company had given me motivation and energy to be here today. I thank to all I've listed down below, and all who I forgot to mention.


I would like to thank my advisor Prof. Dr. Christian Rothenberg for the trust and for letting me be part of his selected group of students. I have to say that I'm extremely grateful for all his patience and understanding all over these last years. Also, his leadership will be a source of inspiration for the rest of my career, that is just beginning. I would not be able to imagine the undertaking of this research without his innovative ideas, consistent support and continuous encouragement. Specially encouragement, since sometimes, especially on research things do not happen as we would expect, and start from the beginning is always a hard task. I would like to express my gratitude honor for having such a great instructor, teacher, leader and friend all for these past years. 


Thanks to all the intrigers, desk, and group colleagues  Alex, Javier, Nathan, Claudio, Daniel, Gyanesh, Raphael, Fabricio and all who I have not mentioned in this text.  Also, thanks to all my LCA colleagues, especially Mijail, Suelen, Amadeu,  Paul, ...

Thanks to all my colleagues and friends I made all these years I've been at Unicamp. 

Thanks to all my Opus Dei friends, especially Priest Fabiano for all his advice, and my friend Denis who have given me a huge support.

Thanks to all my house companions from my old home house P7, and all my "moradia" friends, in particular, my friend(almost brother) Lucas Zorzetti(Xildo).
 
Thanks to my girlfriend Rubia Agondi, for all her support, help, love, understanding and patience, and for making me always believe on my work.

Thanks to my lovely family, my father Tirso José Paschoalon for all his attention and education. To my mother, Rosangela dos Santos Mota, for all her affection and love. And to my sister Ariela Paschoaln, for her company and affection.

And last, and more important, I thank God for all his gifts, protection, and love.

}

%%%%%%%%%%%%%%%%%%%%%%%%%%%%%%%%%%%%%%%%%%%%%%%%%%%%%%%%%%%%%%%%%%%%%%%%%%%%%%%%
\newcommand{\Agradecimentos}{
	exemplo	
}


%%%%%%%%%%%%%%%%%%%%%%%%%%%%%%%%%%%%%%%%%%%%%%%%%%%%%%%%%%%%%%%%%%%%%%%%%%%%%%%%
\newcommand{\Epigrafe}{
\textit{``ratio in homine sicut Deus in mundo''\\
``reason in man is rather like God in the world.''\\
``razão no homem é como Deus no mundo''}
}

%%%%%%%%%%%%%%%%%%%%%%%%%%%%%%%%%%%%%%%%%%%%%%%%%%%%%%%%%%%%%%%%%%%%%%%%%%%%%%%%
\newcommand{\EpigrafeAuthor}{
De regno ad regem Cypri -- Saint Thomas Aquinas (Santo Tomas de Aquino)
}

%%%%%%%%%%%%%%%%%%%%%%%%%%%%%%%%%%%%%%%%%%%%%%%%%%%%%%%%%%%%%%%%%%%%%%%%%%%%%%%%
\newcommand{\Dedicatoria}{

Nessa dedicatória, gostaria de agradecer a todos que me ajudarem por essa etapa, direta ou indiretamente. Aqueles que me inspiraram e me motivaram a seguir por esse caminho, aqueles que me ensinaram e me ajudaram durante o processo, e a aqueles cuja simples companhia me deram energia e me motivaram para estar aqui onde estou hoje. Agradeço a todos, seja os que estão listados abaixo, bem como aqueles cuja minha memória não me ajudou na escrita desse texto.


Gostaria de agradecer ao meu professor e orientador Christian Esteve Rothemberg, sem o qual, seja pelo ensino, seja pela orientação e apoio durante o projeto, este trabalho não teria saído do papel. Gostaria de agradecer toda sua paciência e entendimento por esses últimos anos. Sua liderança será uma fonte de inspiração para mim para o restante de minha carreira, e ela está apenas começando. Sem sua idéias inovadoras, suporte e encorajamento continuo, este projeto não teria saído do papel. Especialmente pelo fato de que em pesquisa muitas vezes as coisas não saem como o esperado, e temos que reiniciar do zero o processo. Eu gostaria de expressar a minha gratisão e honra por ter um tão grande orentador, professor, lider e amigo durante estes anos.


Agradeço também a todos os Intrigers, colegas de grupo e de bancada, Alex, Javier, Nathan, Cláudio, Daniel, Danny, Gyanesh, Rafael, Fabricio e todos os demais que não mencionei neste texto. Agradeço a todos os demais colegas de laboratório do LCA, em especial a Mijail, Suelen, Amadeu, Paul,....


Agradeço a todos os companheiros e amigos que fiz em todos esses anos de Unicamp.


Agradeço a todos os grandes amigos e companheiros da Opus Dei, em especial Padre Fabiano pelos conselhos, e ao meu amigo Denis, grande amigo pelo apoio.


Agradeço aos meus companheiros de minha antiga casa P7, e da moradia, em especial o meu amigo(quase irmão) Lucas Zorzetti(Xildo) . 


Agradeço a minha namorada Rubia Agondi pelo se apoio, ajuda, amor, compreensão e paciência, e por sempre me fazer acreditar em meu trabalho.


Agradeço a minha tão adorada família, a meu Pai Tirso José Paschoalon por todo sua preocupação e ensino. A minha Mãe Rosângela dos Santos Mota, por todo o seu carinho e amor. E a minha irmã Ariela Paschoalon, pela companhia e afeto.


 E por último e mais importante, agradeço a Deus por todos seu dons, proteção e amor. 


}
